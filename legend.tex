\documentclass[a4paper,12pt,landscape]{article}

\usepackage[utf8]{inputenc}
\usepackage[T2A]{fontenc}
\usepackage[english,russian]{babel}
\usepackage[landscape,margin=2cm]{geometry}

\usepackage{graphicx}
\usepackage{color}
\usepackage{indentfirst}
\usepackage{fancyhdr}
\usepackage{needspace}
\usepackage{tikz}

\usetikzlibrary{arrows}
\usetikzlibrary{positioning}

\pagestyle{fancy}
\lhead{Настолка пока что без названия}
\rhead{ЛКШ}
\rfoot{2015, август}

\begin{document}
  \section{Легенда мира}
  Планета X давно страдает от перенаселения. Эльфы, гномы, гоблины, тритоны, оборотни и зиалы теснятся на исторически отведенных им кусочках земли. Исследования решения этой проблемы ведутся повсеместно и безуспешно. Во всяком случае, до этих пор...
~\\  

\textit{''~-- Да, я понял! Вот последний фрагмент кода... ~-- гоблин достал из принтера только что отпечатанные листы.
~-- Не может быть! Покажи... ~-- эльфийка выдернула результаты исследования манускрипта у него из рук. ~-- <<Y содержит в себе энергию, способную разорвать ткань пространства-времени и открыть портал в другой мир>>. Неужели эти отходы производства могут быть полезны? Это прорыв!''}
~\\  

Члены международной исследовательской группы получили множество наград от правителей своих государств. Но все оказалось не так просто...
~\\

\textit{''Зиала подключила последний электрод к куску Y и проверила контакты. Результаты сканирования отобразились на экране. Они были неутешительны. Новый мир был слишком мал. Там не хватало места на всех...''}
~\\

Вы играете за правителя одного из великих государств. Ваша задача: первым открыть портал в другой мир и переселиться туда. В вашем мире магия и техника веками развивались неотделимо. Поэтому у вас есть два варианта активации портала: при помощи энергии ядерного реактора или рунного круга.

  \section{Расы}
  
    \subsection{Оборотни}
    \begin{itemize}
      \item \textbf{Расовая способность:} на четном круге преимущество в выработке стали, на нечетном ~--- мифрила.
      \item Агрессивные и импульсивные. Однажды не поладили с эльфами и получили проклятие.
      \item \textbf{Правители}
        \begin{enumerate}
          \item \textbf{Маликан Вульф}
          
          Захватчик. Узурпатор власти. Загрыз предыдущего.
          
          \textit{Способность персонажа:} Жертвует 1/2 всех ресурсов и получает (количество сброшенных ресурсов)/$k$ карт событий, но не меньше 1.
          
          \item \textbf{Фестор Беар}
          
          Законный вождь. Держит всех в ежовых рукавицах, никто не смеет ему перечить.
          
          \textit{Способность персонажа:} Когда исследует здание может сразу построить один экземпляр без затраты ресурсов, однако оно начнет функционировать только на следующем ходу Фестора.
          
          \item \textbf{Вения Фокс}
          
          Наследница престола, однако не самая старшая. Впрочем, это не помешало ей занять престол, убив двух старших братьев. Хитрая и коварная.       
          
          \textit{Способность персонажа:} Раз в два хода ворует $k$ ресурсов.
          \end{enumerate}
    \end{itemize}
    
    \subsection{Эльфы}
    \begin{itemize}
      \item \textbf{Расовая способность:} преимущество в выработке ауры.
      \item \textbf{Правители} 
        \begin{enumerate}
          \item \textbf{Хиния Кровожадная}
          
          Хладнокровная и расчетливая. Подстроила свержение предыдущего правителя и во время смуты пришла к власти.

          \textit{Способность персонажа:} Способна заблокировать выработку ресурсов одного типа на один круг.
          
          \item \textbf{Лоунд Мудрый}
          
          Аккуратный и вдумчивый правитель. Однако его консерватизм многим приходится не по душе.

          \textit{Способность персонажа:} Вырабатывает на $k$ единиц ауры больше за ход.
          
          \item \textbf{Таниус Ловкий}
          
          Осуществил революцию и пришел к власти. Идеалистичные понятия о мире.

          \textit{Способность персонажа:} Раз в два круга может украсть карту событий из руки любого игрока.
        \end{enumerate}
    \end{itemize}    
    
    \subsection{Гоблины}
    \begin{itemize}
      \item \textbf{Расовая способность:} преимущество в выработке электричества.
      \item \textbf{Правители}
        \begin{enumerate} 
          \item \textbf{Рорш Транден}
          Сумасшедший подрывник, сорви-голова. Взорвал замок и начал сове правление на его руинах.

          \textit{Способность персонажа:} Раз в два круга может сбросить карту событий и взорвать одно здание одного игрока.
          
          \item \textbf{Гриндо Фалман}
          
          Брат Занги, пришли к власти одновременно, правят вместе. За счет этого принимают взвешенные решения. Предпочитает дипломатические пути решения.
          \textit{Способность персонажа:} 
          
          \item \textbf{Занга Фалман}
          
          Сестра Гриндо. Сторонница технического прогресса.

          \textit{Способность персонажа:} Может тратить на $1$ единицу ресурсов меньше при постройке за один ход.
        \end{enumerate}
    \end{itemize}
    
    \subsection{Гномы}
    \begin{itemize}
      \item \textbf{Расовая способность:} преимущество в выработке стали.

      \item \textbf{Правители}
        \begin{enumerate}
          \item \textbf{Халльмунд}
          Выходец из рода купцов. Безмерно богат. Свое государство он попросту купил.

          \textit{Способность персонажа:} Не требуется рынок, чтобы торговать.
          
          \item \textbf{Ингрид}
          Женщина-воин. Объединила государство после смуты.
          
          \textit{Способность персонажа:} Во время глобальных отрицательных событий, связанных с потерей ресурсов, теряет на 1 ресурс меньше.
          
          \item \textbf{Торвард}
          Честно избранный правитель.
          
          \textit{Способность персонажа:}
        \end{enumerate}
    \end{itemize}
    
    \subsection{Тритоны}
    \begin{itemize}
      \item \textbf{Расовая способность:} преимущество в выработке мифрила
      \item \textbf{Правители}
        \begin{enumerate}
          \item \textbf{Тако} 
        
          На самом деле, настоящий Тако давно мертв. Вместо него правит самозванец, в совершенстве овладевший искусством маскировки.

          \textit{Способность персонажа:} Скинув какой-либо ресурс может перенаправить событие, направленное на него.
          
          \item \textbf{Сакана}
          Основательница государства тритонов.
          
          \textit{Способность персонажа:} В начале хода может скинуть всю руку и взять столько же карт из колоды.
          
          \item \textbf{Имори}
          
          Никто не помнит его имени, никто не помнит, как он пришел к власти. Впрочем, он сам тоже не помнит.
          
          \textit{Способность персонажа:} Сбросив 2(3) карты событий, может заставить другого игрока на один ход забыть какое-либо исследование.
          
        
      \end{enumerate}
    \end{itemize}
    
    \subsection{Зиалы}
    \begin {itemize}
      \item \textbf{Расовая способность:} 
      \item Самая древняя из рас. Образовались за счет всплеска магии при создании мира.
      \item \textbf{Правители}
        \begin{enumerate}
          \item \textbf{Аглая}
          
          Целеустремленная, жадная до знаний. Однако о портале и о том, что за другой его стороной, ей ничего не известно. Информация об этом стала ее целью, ее манией.
          
          \textit{Способность персонажа:} Раз в два хода может использовать одно из исследований не выше 3-го ранга так, как будто оно уже изучено.
          
          \item \textbf{Даратас}
          
          Зиалы известны своим талантом к ментальным воздействиям, поэтому у них в государстве больше распространены восстания не вооруженные. Группа повстанцев день за днем посылает ментальные волны в мозг их правителя ~--- Даратаса. Он чувствует, что сходит с ума. Неконтролируемые вспышки невменяемости случаются все чаще. Он хочет спасти свой народ, пока окончательно не сошел с ума.
          
          \textit{Способность персонажа:} Раз в два хода способен запретить прокачку любого исследования не выше 3-го ранга, заплатив его стоимость.
          
          \item \textbf{Катан}
          
          Провидец. Ему было известно, когда он станет правителем и как именно, он видел свою смерть... И чтобы ее избежать, он должен провести свой народ через портал. Тогда его судьба изменится...
          
          \textit{Способность персонажа:} Раз в ход может посмотреть верхнюю карту колоды событий и или вернуть ее на то же место, или положить под низ колоды.
          
        \end{enumerate}
        
    \end{itemize}
\end{document}
