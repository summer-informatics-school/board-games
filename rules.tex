\documentclass[a4paper,12pt]{article}

\usepackage[utf8]{inputenc}
\usepackage[T2A]{fontenc}
\usepackage[english,russian]{babel}

\usepackage{graphicx}
\usepackage{color}
\usepackage{indentfirst}
\usepackage{fancyhdr}
\usepackage{needspace}

\pagestyle{fancy}
\lhead{Настолка пока что без названия}
\rhead{ЛКШ}
\rfoot{2015, август}

\begin{document}
  \section{Правила}

    \subsection{Легенда}
      Мир, населенный шестью разными расами, страдает от перенаселения. Каким-то образом у каждой расы обнаруживается камень, активация которого может открыть портал в другой мир. Однако места там только на одну расу. Кто успеет, тот и спасется.
    \subsection{Состав игры}
    	\begin{enumerate}
    	  \item Карта изобретений и построек(одна на всех? каждому игроку?)
    	    \begin{itemize}
             \item Какие будут изобретения? 
             \item Как они будут взаимодействовать между собой?
           \end{itemize}  
    	  	
	  \item Фишки для изобретений и построек
	  \item Карточки персонажей (Видимо по 2-3 типа для каждой расы)
	  \item Карточки ресурсов
	  \item Карточки действий
	    \begin{itemize}
    	      \item Проклятия
    	      \item Бонусы
    	      \item Что еще?
          \end{itemize}
	\end{enumerate}


    \subsection{Подготовка игры}
    	\begin{enumerate}
    	  \item Каждый выбирает расу за которую будет играть.
    	  \item Каждый рандомно вытягивает правителя совей расы.
    	  \item Раздать начальные ресурсы
    	  \item Раздать фишки для изобретений
    	  \item Определить, кто первый ходит(Можно что-нибудь оригинальное как в ханаби или TTR)
	  \item Что-нибудь еще?
    	\end{enumerate}

    \subsection{Ход игры}

      Игра состоит из нескольких ходов, выполняемых по очереди.

      Ход одного игрока состоит из следующих действий:
      \begin{enumerate}
        \item Производство ресурсов
        \item Взять очередную карту из колоды событий.
          Все игроки, начиная с ходящего, выполняют коллективное
          действие, описанное на ней. Карта отправляется в колоду
          сброса.
        \item Взять себе в руку взакрытую очередную карту из колоды.
        \item Можно сыграть одну или несколько карт из руки. 
        \item Можно произвести одно или несколько исследований.
      \end{enumerate}

    \subsection{Завершение игры}

      Игра завершается, когда один из игроков на своем ходу
      открывает портал. % изобретает трактор

    \subsection{Карты событий}

      Каждая карта включает в себя:
      \begin{itemize}
        \item Название
        \item Легенду
        \item Действие на всех игроков
        \item Действие на одного игрока
        \item Стоимость действия на одного игрока
      \end{itemize}

    \subsection{Характеристики игроков}
      \begin{enumerate}
        \item Начальные ресурсы
        \item ''Предрасположенность'' к магии или технике
        \item Скорость постройки изобретений?
        \item Количество начальных карт?
        \item Количество набираемых карт?
        \item Активируемая способность вроде взять n карт взакрытую, выбрать одну, остальное сбросить?
      \end{enumerate}

    \subsection{Расы}

      Эльфы, гномы, хоббиты, гоблины, орки, летучие каракатицы, (русалки, оборотни, вампиры). Какие у них могут быть особенности?

    \subsection{Исследования}
\end{document}
