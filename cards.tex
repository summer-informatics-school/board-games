\documentclass[a4paper,12pt]{article}

\usepackage[utf8]{inputenc}
\usepackage[T2A]{fontenc}
\usepackage[english,russian]{babel}

\usepackage{graphicx}
\usepackage{color}
\usepackage{indentfirst}
\usepackage{fancyhdr}
\usepackage{needspace}
\usepackage{tikz}
\usepackage[lmargin=1cm,rmargin=1cm]{geometry}

\usetikzlibrary{arrows}
\usetikzlibrary{positioning}

\pagestyle{fancy}
\lhead{Настолка пока что без названия}
\rhead{ЛКШ}
\rfoot{2015, август}

\begin{document}
  \begin{tabular}{|p{5cm}|p{5cm}|p{5cm}|}
    \hline
      \textbf{???} & \textbf{???} & \textbf{???} \\
      Возьмите еще одну карту событий & Возьмите еще одну карту событий & Возьмите еще одну карту событий \\
      Сталь: 1 & Сталь: 1 & Сталь: 1 \\
    \hline
      \textbf{???} & \textbf{???} & \textbf{???} \\
      Возьмите еще одну карту событий & Возьмите еще одну карту событий & Возьмите еще одну карту событий \\
      Сталь: 1 & Сталь: 1 & Сталь: 1 \\
    \hline
      \textbf{???} & \textbf{???} & \textbf{???} \\
      Возьмите еще одну карту событий & Возьмите еще одну карту событий & Возьмите еще одну карту событий \\
      Сталь: 1 & Сталь: 1 & Сталь: 1 \\
    \hline
      \textbf{???} & \textbf{???} & \textbf{???} \\
      Возьмите еще одну карту событий & Возьмите еще одну карту событий & Возьмите еще одну карту событий \\
      Сталь: 1 & Сталь: 1 & Сталь: 1 \\

    \hline
      \textbf{???} & \textbf{???} & \textbf{???} \\
      Возьмите еще две карты событий & Возьмите еще две карты событий & Возьмите еще две карты событий \\
      Энергия: 1 & Энергия: 1 & Энергия: 1 \\
    \hline
  \end{tabular}
  \newpage
  \begin{tabular}{|p{5cm}|p{5cm}|p{5cm}|}
    \hline
      \textbf{???} & \textbf{???} & \textbf{???} \\
      Возьмите еще две карты событий & Возьмите еще две карты событий & Возьмите еще две карты событий \\
      Энергия: 1 & Энергия: 1 & Энергия: 1 \\

    \hline
      \textbf{???} & \textbf{???} & \textbf{???} \\
      Возьмите еще три карты событий & Возьмите еще три карты событий & Возьмите еще три карты событий \\
      Аура: 1, энергия: 1 & Аура: 1, энергия: 1 & Аура: 1, энергия: 1 \\

    \hline
      \textbf{Горный горизонт} & \textbf{Горный горизонт} & \textbf{Горный горизонт} \\
      Получите 3 карты стали & Получите 3 карты стали & Получите 3 карты стали \\

    \hline
      \textbf{Горный горизонт} & \textbf{Горный горизонт} & \textbf{Горный горизонт} \\
      Получите 3 карты стали & Получите 3 карты стали & Получите 3 карты стали \\

    \hline
      \textbf{Горный горизонт} & \textbf{Горный горизонт} & \textbf{Горный горизонт} \\
      Получите 3 карты мифрила & Получите 3 карты мифрила & Получите 3 карты мифрила \\
    \hline
  \end{tabular}
  \newpage
  \begin{tabular}{|p{5cm}|p{5cm}|p{5cm}|}

    \hline
      \textbf{Горный горизонт} & \textbf{Горный горизонт} & \textbf{Горный горизонт} \\
      Получите 3 карты мифрила & Получите 3 карты мифрила & Получите 3 карты мифрила \\

    \hline
      \textbf{Выброс ауры} & \textbf{Выброс ауры} & \textbf{Выброс ауры} \\
      Получите 3 карты ауры & Получите 3 карты ауры & Получите 3 карты ауры \\

    \hline
      \textbf{Солнечный ветер} & \textbf{Солнечный ветер} & \textbf{Солнечный ветер} \\
      Получите 3 карты энергии & Получите 3 карты энергии & Получите 3 карты энергии \\
    \hline
  \end{tabular}

  \begin{enumerate}
    \item Возьмите одну карту в руку (за 1~ед. стали)
    \item Возьмите две карты в руку (за 1~ед. электроэнергии)
    \item Возьмите три карты в руку (за 1~ед. ауры + 1~ед. электроэнергии)
    \item Получите 3~ед. стали
    \item Получите 3~ед. мифрила
    \item Получите 3~ед. ауры
    \item Получите 3~ед. электроэнергии
    \item Потеряйте 2~ед. электроэнергии
    \item Потеряйте 2~ед. стали
    \item Потеряйте 2~ед. мифрила
    \item Потеряйте 2~ед. ауры
    \item Если у вас не наименьшее количество мифрила, раздайте половину любым игрокам. Округление вниз.
    \item Если у вас не наименьшее количество стали, раздайте половину любым игрокам. Округление вниз.
    \item Если у вас не наименьшее количество электроэнергии, раздайте половину любым игрокам. Округление вниз.
    \item Если у вас не наименьшее количество ауры, раздайте половину любым игрокам. Округление вниз.
    \item Одно ваше строение на ваш выбор будет разрушено.
    \item Возьмите карту из сброса на ваш выбор.
    \item Посмотрите карты любого другого игрока.
    \item Сбросьте половину своих карт. Округление вниз.
    \item Вы можете взять любое количество карт из колоды по две любых ресурса за карту.
    \item ...
  \end{enumerate}

\end{document}
